\input{/home/vysotskiial/study/tex_include/LectSlidesHeader}
\setbeamertemplate{caption}[numbered]
\title[]{Наблюдатели для систем с неопределенностью при наличии неидеальностей в релейных элементах.}
\subtitle{}

\author[Высоцкий А.О., Фомичев В.В.]
{Высоцкий Алексей Олегович \\
Фомичев Василий Владимирович}

\institute[]
{
МГУ имени М.В. Ломоносова\\
Факультет вычислительной математики и кибернетики\\
Кафедра нелинейных динамических систем и процессов управления
}

\newtheorem{statement}{Утверждение}

\begin{document}

\begin{frame}[t]
	\maketitle
\end{frame}

\begin{frame}
	\frametitle{Постановка задачи}

	Пусть задана система
	\begin{equation}\label{base_sys}
		\begin{cases}
			\dot{x} = Ax + Bu + B'\xi\\
			y = Cx,
		\end{cases}
	\end{equation}
	где $x \in \mathbb{R}^n$ -- неизвестный фазовый вектор, $u \in \mathbb{R}$ --
	известный вход (управление), $\xi \in \mathbb{R}$ -- неизвестный вход
	(возмущение), $y \in \mathbb{R}$ -- измеряемый выход системы, стационарные
	матрицы $A$, $B$, $B'$ и $C$ -- известные. Требуется по известной информации об
	управлении $u(t)$, выходе $y(t)$ и параметрах системы построить асимптотическую
	(при $t \rightarrow +\infty$) оценку фазового вектора.
\end{frame}

\begin{frame}
	\frametitle{Основные предположения}
	\begin{assumption}\label{ass_gen}
		Система \eqref{base_sys} находится в общем положении.
	\end{assumption}
	\begin{assumption}\label{ass_rel_or}
		Относительный порядок системы равен $r>1$.
	\end{assumption}
	\begin{assumption}\label{ass_null_dyn}
		Инвариантные нули системы устойчивы (т.е. $Re(s^*)<0$, где $s^*$ --
		инвариантный ноль системы).
	\end{assumption}
	\begin{assumption}\label{ass_bound_input}
		Неизвестный вход (помеха) кусочно-непрерывен и равномерно ограничен, причем
		известна его мажоранта, т.е. $|\xi(t)| \leq \xi_0$.
	\end{assumption}
\end{frame}

\begin{frame}
	\frametitle{Преобразованная система}
	Можно показать, что при выполнении этих предположений исходную задачу можно
	свести к задаче наблюдения для системы
	\begin{equation}
		\begin{cases}
			\dot{\varepsilon} = \tilde{A}\varepsilon + b_0\xi\\
			\tilde{y} = \varepsilon_1,
		\end{cases}
		\label{final_system}
	\end{equation}
	где
	\begin{equation*}
		\tilde{A} =
		\begin{pmatrix}
			-l_1     & 1      & 0      & \dots  & 0     \\
			-l_2     & 0      & 1      & \dots  & 0     \\
			\vdots           & \vdots & \vdots & \vdots & \vdots\\
			-l_{r-1} & 0      & 0      & \dots  & 1     \\
			-l_r     & 0      & 0      & \dots  & 0
		\end{pmatrix},
		b_0 =
		\begin{pmatrix}
			0\\
			0\\
			\vdots \\
			0\\
			1
		\end{pmatrix},
	\end{equation*}
	коэффициенты $l_i$ таковы, что матрица $\tilde{A}$ -- гурвицева.
\end{frame}

\begin{frame}
	\frametitle{Вид наблюдателя}

	Для восстановления же вектора состояния системы \eqref{final_system} предлагалось
	использовать каскад систем вида

	\begin{equation}
		\begin{cases}
			\dot{\tilde{\varepsilon}}_i = -l_i\tilde{y} + \bar{\varepsilon}_{i+1} +
			sgn(\bar{\varepsilon}_i - \tilde{\varepsilon}_i) |\bar{\varepsilon}_i -
			\tilde{\varepsilon}_i|^\frac{1}{2}\\
			\dot{\bar{\varepsilon}}_{i+1} = -l_{i+1}\tilde y + \mu_i sgn(\bar{\varepsilon}_i
			- \tilde{\varepsilon}_i),
		\end{cases},
		i = 1, \dots, r - 1
		\label{obs_cascade}
	\end{equation}
	где $\bar{\varepsilon}_i(t)$ -- оценка сигнала $\varepsilon_i(t)$, построенная на
	предыдущем шаге; $\bar{\varepsilon}_{i+1}(t)$ -- оценка сигнала
	$\varepsilon_{i+1}(t)$, вырабатываемая наблюдателем \eqref{obs_cascade} на
	текущем шаге. В качестве оценки $\bar{\varepsilon}_1$ используется выход
	$\tilde{y}$ системы \eqref{final_system}.
\end{frame}

\begin{frame}
	\frametitle{Система в отклонениях}
	При таком способе построения наблюдателя система в отклонениях для каждой из
	систем каскада \eqref{obs_cascade} будет иметь вид

	\begin{equation}
		\begin{cases}
			\dot{e}_i = e_{i + 1} - sgn(e_i + \delta) \sqrt{\vert e_i + \delta \vert}\\
			\dot{e}_{i+1} = -\mu_i sgn(e_1 + \delta) + \xi_i\\
		\end{cases},
		\label{base_sys}
	\end{equation}
	где $\delta = \bar{\varepsilon}_i - \varepsilon_i$.

	Было доказано, что если $|\delta| \le \Delta = const$, то траектория системы
	\eqref{base_sys} сходится в область, размеры которой не превышают $F(\Delta)$,
	конкретнее
	\begin{align*}
		&|e_2| \leq max((\frac{1}{\sqrt{\nu}-1}(C_1 \Delta + C_2 \Delta^2)(\mu
		+ \xi_0))^{\frac{1}{2}}; \sqrt{\Delta}) = F_2(\Delta),\\
		&|e_1| \leq F_2(\Delta) + \Delta = F_1(\Delta),
	\end{align*}
\end{frame}

\begin{frame}
	\frametitle{Зона нечувствительности}
	Пусть вместо идеальных реле в системе \eqref{base_sys} имеются реле следующего вида
	\begin{equation}
		sgn_{ins}(x) =
		\begin{cases}
			1,\; x > \Delta_{ins},\\
			-1,\; x < -\Delta_{ins},\\
			0,\; \vert x \vert \le \Delta_{ins}
		\end{cases},
		\label{insens_relay}
	\end{equation}
	\begin{statement}
		В случае наличия неидеальности вида \eqref{insens_relay} в системе
		\eqref{base_sys} траектория системы будет сходится в область, размер которой не
		превышает $F(\Delta + \Delta_{ins})$
	\end{statement}
\end{frame}

\begin{frame}
	\frametitle{Гистерезис}
	Пусть вместо идеальных реле в системе \eqref{base_sys} имеются реле с
	неидеальностью типа гистерезис, то есть:
	\begin{equation}
		sgn_h(x(t)) =
		\begin{cases}
			1,\; x > \Delta_h,\\
			-1,\; x < -\Delta_h,\\
			sgn(x(\tau(t))),\; \vert x \vert \le \Delta_h
		\end{cases},
		\label{hyst_relay}
	\end{equation}
	где $\tau(t) = \sup{\{\tau \le t: |x(\tau)| = 1\}}$.
	\begin{statement}
		В случае наличия неидеальности вида \eqref{hyst_relay} в системе
		\eqref{base_sys} траектория системы будет сходится в область, размер которой не
		превышает $F(\Delta + \Delta_{h} + \varepsilon)$, где $\varepsilon > 0$ --
		любое, сколь угодно малое число.
	\end{statement}
\end{frame}

\begin{frame}
	\frametitle{Задержка}
	Пусть в системе \eqref{base_sys} имеется задержка в элементах переключения, т.е.
	\begin{equation}
		sgn_{\tau}(x(t)) = sgn(x(t-\tau))
		\label{delay_relay}
	\end{equation}
	\begin{statement}
		В случае наличия неидеальности вида \eqref{delay_relay} в системе
		\eqref{base_sys} в установившемся режиме будет справедливо неравенство
		\begin{equation*}
			e_2^* \ge \frac{\tau\sqrt{\nu}(\mu - \xi)}{1 - \sqrt{\nu}},
		\end{equation*}
		где $e_2^*$ -- координаты пересечения траекторией системы оси $e_1 = 0$.
	\end{statement}
\end{frame}
\end{document}
